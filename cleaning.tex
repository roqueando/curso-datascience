% Created 2024-08-04 Sun 00:53
% Intended LaTeX compiler: pdflatex
\documentclass[presentation]{beamer}
\usepackage[utf8]{inputenc}
\usepackage[T1]{fontenc}
\usepackage{graphicx}
\usepackage{longtable}
\usepackage{wrapfig}
\usepackage{rotating}
\usepackage[normalem]{ulem}
\usepackage{amsmath}
\usepackage{amssymb}
\usepackage{capt-of}
\usepackage{hyperref}
\usetheme{default}
\author{Omolu}
\date{\today}
\title{Limpeza de Dados com Python}
\hypersetup{
 pdfauthor={Omolu},
 pdftitle={Limpeza de Dados com Python},
 pdfkeywords={},
 pdfsubject={},
 pdfcreator={Emacs 29.4 (Org mode 9.7.9)}, 
 pdflang={English}}
\begin{document}

\maketitle
\begin{frame}{Outline}
\tableofcontents
\end{frame}

\begin{frame}[label={sec:org6b1fb3e},fragile]{Analise de Churn}
 \begin{block}{Importacao e criacao do dataframe}
\begin{verbatim}
import pandas as pd
df = pd.read_csv('cleaning/Churn.csv', sep=';')
print(df)
\end{verbatim}
\end{block}
\begin{block}{Renomear as colunas}
\begin{verbatim}
df.columns = ["id", "score", "state", "gender", "age", "qty_patrimony", "balance", "qty_products", "has_credit_card", "is_active", "salary", "churn"]
print(df)
\end{verbatim}
\begin{block}{Dados categoricos}
\begin{itemize}
\item State
\item Gender
\end{itemize}
\end{block}
\end{block}
\begin{block}{Exploracao dos dados categoricos}
\begin{block}{Dados categoricos sao os dados que sao nao-calculaveis, podendo ser tambem ordenaveis}
\end{block}
\begin{block}{Categoricos Nominais ou Categoricos Ordinais}
\end{block}
\begin{block}{Estado}
\begin{verbatim}
state = df.groupby(["state"]).size()
state.plot.bar()
\end{verbatim}
\end{block}
\begin{block}{Genero}
\begin{verbatim}
gender = df.groupby(["gender"]).size()
gender.plot.bar()
\end{verbatim}
\end{block}
\end{block}
\begin{block}{Exploracao dos dados numericos}
\begin{block}{score}
\begin{block}{describe}
\begin{verbatim}
print(df['score'].describe())
\end{verbatim}
\end{block}
\begin{block}{boxplot}
\begin{verbatim}
import seaborn as sns
sns.boxplot(df['score']).set_title('score')
\end{verbatim}
\end{block}
\begin{block}{histograma}
\begin{verbatim}
sns.distplot(df['score']).set_title('score')
\end{verbatim}
\end{block}
\end{block}
\begin{block}{age}
\begin{block}{describe}
\begin{verbatim}
print(df['age'].describe())
\end{verbatim}
\end{block}
\begin{block}{boxplot}
\begin{verbatim}
sns.boxplot(df['age']).set_title('age')
\end{verbatim}
\end{block}
\begin{block}{histogram}
\begin{verbatim}
sns.distplot(df['age']).set_title('age')
\end{verbatim}
\end{block}
\end{block}
\begin{block}{balance}
\begin{block}{describe}
\begin{verbatim}
print(df['balance'].describe())
\end{verbatim}
\end{block}
\begin{block}{boxplot}
\begin{verbatim}
sns.boxplot(df['balance']).set_title("balance")
\end{verbatim}
\end{block}
\begin{block}{histograma}
\begin{verbatim}
sns.distplot(df['balance']).set_title('balance')
\end{verbatim}
\end{block}
\end{block}
\begin{block}{salary}
\begin{block}{describe}
\begin{verbatim}
print(df['salary'].describe())
\end{verbatim}
\end{block}
\begin{block}{boxplot}
\begin{verbatim}
sns.boxplot(df['salary']).set_title('salary')
\end{verbatim}
\end{block}
\begin{block}{histograma}
\begin{verbatim}
sns.distplot(df['salary']).set_title('salary')
\end{verbatim}
\end{block}
\end{block}
\end{block}
\begin{block}{checando se ha valores nulos}
\begin{verbatim}
print(df.isnull().sum())
\end{verbatim}
\end{block}
\begin{block}{Tratamento dos dados categoricos}
\begin{block}{state}
\begin{verbatim}
t = df.loc[df['state'] != 'RP']
t = t.loc[df['state'] != 'TD']
t = t.loc[df['state'] != 'SP']
print(t.groupby(['state']).size())
\end{verbatim}
\end{block}
\begin{block}{gender}
\begin{verbatim}
#df['gender'].fillna('Masculino', inplace=True)
df.loc[df['gender'] == 'M', 'gender'] = 'Masculino'
df.loc[df['gender'] == 'F', 'gender'] = 'Feminino'
df.loc[df['gender'] == 'Fem', 'gender'] = 'Feminino'
print(df)
\end{verbatim}

\begin{verbatim}
print(df['gender'].value_counts())

\end{verbatim}
\end{block}
\end{block}
\begin{block}{Tratamento dos dados numericos}
\begin{block}{score}
nada pra fazer, nao ha score 0 nem acima de 1000
\end{block}
\begin{block}{age}
\begin{itemize}
\item Se seria uma instituicao financeira, logo apenas pessoas acima de 18 anos deveria estar aqui
porem como ha de -20 a 140, e pode ocorrer de ter linhas com dados importantes, devemos apenas
fazer a media das idades para `age <= 18 \&\& age >= 80` tendo em mente que o range de idade
seria de 18 a 80 anos
\end{itemize}
\end{block}
\end{block}
\end{frame}
\end{document}
